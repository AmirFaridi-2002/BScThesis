\documentclass[11pt]{article}

\usepackage[a4paper,margin=1in]{geometry}
\usepackage{amsmath,amssymb,amsthm,mathtools,mathrsfs}
\usepackage{braket,stmaryrd}
\usepackage{enumitem}
\usepackage{xcolor}
\usepackage{tikz-cd}

\newcommand{\LL}{\mathcal{L}}
\newcommand{\Tr}{\mathrm{tr}}
\newcommand{\Id}{\mathbf{1}}
\newcommand{\QO}{\mathrm{QO}}
\newcommand{\sq}{\sqsubseteq}
\newcommand{\den}[1]{\llbracket #1 \rrbracket}
\newcommand{\Var}{\mathsf{Var}}
\newcommand{\Val}{\mathsf{Val}}
\newcommand{\AExp}{\mathsf{AExp}}
\newcommand{\BExp}{\mathsf{BExp}}
\newcommand{\Cmd}{\mathsf{Cmd}}
\newcommand{\post}{\operatorname{post}}
\newcommand{\wlp}{\operatorname{wlp}}
\newcommand{\Guard}{\operatorname{Guard}}
\newcommand{\lfp}{\operatorname{lfp}}
\newcommand{\supp}{\operatorname{supp}}
\newcommand{\ok}{\mathsf{ok}}
\newcommand{\er}{\mathsf{er}}
\newcommand{\true}{\mathsf{true}}
\newcommand{\false}{\mathsf{false}}
\newcommand{\error}{\mathsf{error}}
\newcommand{\abort}{\mathsf{abort}}
\newcommand{\assume}{\mathsf{assume}}
\newcommand{\assert}{\mathsf{assert}}
\newcommand{\ifthenelse}[3]{\mathsf{if}\ #1\ \mathsf{then}\ #2\ \mathsf{else}\ #3}
\newcommand{\whiledo}[2]{\mathsf{while}\ #1\ \mathsf{do}\ #2}
\newcommand{\nat}{\mathsf{nat}}
\newcommand{\infer}[3][]{\frac{#2}{#3}\,#1}
\newcommand{\Skip}{\mathsf{skip}}
\newcommand{\States}{\Sigma}
\newcommand{\Pow}{\mathscr{P}}

\renewcommand{\braket}[2]{\langle #1 | #2 \rangle}

\newcommand{\Reg}{\mathsf{Reg}}
\newcommand{\Done}{\downarrow}
\newcommand{\Conf}[2]{\langle #1,\, #2\rangle}
\newcommand{\step}[1]{\xrightarrow{#1}}
\newcommand{\ptr}{\operatorname{tr}}
\newcommand{\cyl}{\operatorname{cyl}}
\newcommand{\Dens}{\mathcal{D}}
\newcommand{\Dsub}[1]{\mathcal{D}_{\le 1}(#1)}
\newcommand{\comp}[1]{\overline{#1}}


\theoremstyle{plain}
\newtheorem{theorem}{Theorem}[section]
\newtheorem{lemma}[theorem]{Lemma}
\newtheorem{claim}[theorem]{Claim}

\theoremstyle{definition}
\newtheorem{definition}[theorem]{Definition}

\theoremstyle{remark}
\newtheorem{remark}[theorem]{Remark}

\begin{document}

\begin{center}
{\Large Quantum Programming Languages and Semantics}\\[0.5em]
\end{center}
\setcounter{section}{5}



\section{Denotational semantics as super-operators}
\label{sec:coqq-denotational}

This section presents a denotational (compositional) semantics for qwhile programs in the
style used by CoqQ: each command denotes a \emph{super-operator}, i.e.\ a linear map on
operators. The key idea is that we interpret each program as a transformer of (subnormalized)
density operators on the \emph{fixed} global Hilbert space
$ H \;\cong\; \bigotimes_{x\in\Reg} H_x $ introduced earlier.



\paragraph{Super-operators and quantum operations.}
Let $L(H)$ be the complex vector space of linear operators on $H$.
Recall that a \emph{super-operator} on $H$ is a linear map
$ \mathcal{E}: L(H) \to L(H). $
Operationally, we will apply such maps to program states, i.e.\ to partial density operators
\[
  \Dsub{H} \;:=\; \{\rho\in L(H):\ \rho\sqsupseteq 0\ \wedge\ \Tr(\rho)\le 1\}.
\]
The physically meaningful state transformers are those that are \emph{completely positive}
and \emph{trace-nonincreasing}; these are commonly called \emph{quantum operations}.
(Equivalently, they are exactly the maps admitting a Kraus form
$\mathcal{E}(\rho)=\sum_i E_i \rho E_i^\dagger$ with $\sum_i E_i^\dagger E_i \sq \Id$.)
We will define denotations as super-operators and then prove that the denotations are in fact quantum operations.

We write $\den{C}$ for the denotation of a qwhile command $C$.
Formally, $\den{C}$ is a super-operator on $H$:
\[
  \den{C}: L(H)\to L(H),
\]
and it restricts to a map $\Dsub{H}\to \Dsub{H}$.

Primitive commands mention a subsystem $s\subseteq\Reg$ and act only on $H_s$, leaving
$H_{\comp{s}}$ untouched. Denotationally, this is implemented using cylindrical extension.


The denotation is defined by structural recursion on the syntax of $C$. Here we will use the same $\Cmd$ as defined 
in the previous section, with a minor modification: we will use $\abort$ instead of $\error$, since abnormal terminations are not our focus here.
\paragraph{Skip and Abort.}
$
  \den{\Skip}(\rho) \;:=\; \rho,
  \qquad
  \den{\abort}(\rho) \;:=\; 0.
$ \\
Thus $\Skip$ is the identity super-operator, while $\abort$ discards all probability mass and produces
no output state: once $\abort$ is reached, the computation does not return a normal
post-state, so it contributes $0$ to the overall (subnormalized) output.

\paragraph{Sequencing.}
Sequential composition is interpreted by function composition in the expected execution order:
$
  \den{C_1;C_2} \;:=\; \den{C_2}\circ \den{C_1},
  \qquad\text{equivalently}\qquad
  \den{C_1;C_2}(\rho) \;=\; \den{C_2}\!\bigl(\den{C_1}(\rho)\bigr).
$ \\
This clause formalizes that the output of $C_1$ becomes the input of $C_2$.
Because each $\den{C}$ is linear on operators, $\den{C_1;C_2}$ is linear as well.

\paragraph{Initialization.}
For $\mathsf{init}\ \rho_s$, recall the reset operation
$
  \operatorname{Reset}_{s,\rho_s}(\rho) \;:=\; \rho_s \otimes \Tr_s(\rho).
$
Denotationally,
\[
  \den{\mathsf{init}\ \rho_s}(\rho)
  \;:=\;
  \rho_s \otimes \Tr_s(\rho).
\]
This captures that the old contents of subsystem $s$ are discarded (via partial trace), any prior
entanglement across the cut $(s,\comp{s})$ is destroyed, and a fresh local state $\rho_s$ is prepared.

\paragraph{Unitary application.}
For $\mathsf{apply}\ U_s$ with $U_s$ unitary on $H_s$,
\[
  \den{\mathsf{apply}\ U_s}(\rho)
  \;:=\;
  U_s^{(s)}\,\rho\,\bigl(U_s^{(s)}\bigr)^\dagger.
\]
This is conjugation by the cylindrically-extended unitary, i.e.\ ``apply $U_s$ locally on $s$ and do
nothing to $\comp{s}$.''

\paragraph{Conditionals.}
Let $M_s=\{(m,M_m)\}_{m\in\mathsf{Out}(M_s)}$ be a measurement on subsystem $s$ in Kraus form
(with $\sum_m M_m^\dagger M_m=\Id_{H_s}$).
The command
\[
  \mathsf{if}\ (\square m.\ M_s=m\rightarrow C_m)\ \mathsf{fi}
\]
measures $s$ and then executes $C_m$ on the corresponding post-measurement state.
Denotationally, because the measurement outcome is not returned as an explicit classical value,
the overall output is the \emph{sum} (mixture) of the branch outputs:
\[
  \den{\mathsf{if}\ (\square m.\ M_s=m\rightarrow C_m)\ \mathsf{fi}}(\rho)
  \;:=\;
  \sum_{m\in\mathsf{Out}(M_s)}
  \den{C_m}\!\Bigl(M_m^{(s)}\,\rho\,\bigl(M_m^{(s)}\bigr)^\dagger\Bigr) = \den{C_m}\mathcal{I}(\rho).
\]
Each branch state $M_m^{(s)}\rho(M_m^{(s)})^\dagger$ is subnormalized; its trace equals the probability
mass of obtaining outcome $m$ from the current partial state $\rho$.
Summing over $m$ therefore yields a single partial density operator that represents the full ensemble
after the conditional.

\subsection{While semantics: syntactic approximants and limits}
\label{subsec:while-approximants}

A while-loop can, in general, iterate an unbounded number of times, so its semantics cannot be given
by a finite structural recursion. CoqQ resolves this by defining the loop as the limit of an
increasing sequence of \emph{finite unrollings} (syntactic approximants). This construction is also
standard in denotational accounts of classical while-languages.

\paragraph{The loop and its measurement interface.}
Fix a loop command
\[
  W \;\equiv\; \mathsf{while}\ M'_s = 1\ \mathsf{do}\ C\ \mathsf{od},
\]
where $M'_s=\{M_0,M_1\}$ is a two-outcome measurement on $H_s$ (so $M_0^\dagger M_0 + M_1^\dagger M_1=\Id_{H_s}$).
Outcome $0$ terminates the loop; outcome $1$ executes the body $C$ and repeats.

\paragraph{Syntactic approximants.}
Define commands $W^{(n)}$ for integers $n\ge 0$ by:
\[
  W^{(0)} \;:=\; \mathsf{abort},
\]
and
\[
  W^{(n+1)}
  \;:=\;
  \mathsf{if}\ (\square b.\ M'_s=b \rightarrow D_b)\ \mathsf{fi},
  \qquad\text{where}\qquad
  D_0 := \Skip,\quad D_1 := C;W^{(n)}.
\]
Intuitively, $W^{(n)}$ represents ``run the loop for at most $n$ continue-iterations'':
if we observe $n$ consecutive outcomes $1$, the approximant forces nontermination by falling back to
$\mathsf{abort}$. Thus, $W^{(n)}$ captures exactly the contribution of executions that terminate
within $n$ loop-iterations.

Recall the Löwner order $\sq$ on Hermitian operators: $A\sq B$ iff $B-A\sqsupseteq 0$.
As defined earlier, it extends pointwise to super-operators by
\[
  \mathcal{E}\sq \mathcal{F}
  \quad\!\!\iff\quad
  \forall\rho\sqsupseteq 0,\ \mathcal{E}(\rho)\sq \mathcal{F}(\rho).
\]
The approximants form an increasing chain in this order. Concretely, one can prove 
for every $n \ge 0$ and every $\rho\in\Dsub{H}$ that $\den{W^{(n)}}(\rho)\sq \den{W^{(n+1)}}(\rho)$.

The intuitive reason is simple: allowing one more unrolling can only add more terminating behaviors,
hence more positive output mass. Moreover, each $\den{W^{(n)}}(\rho)$ is bounded above by $\rho$ in
trace (indeed $\Tr(\den{W^{(n)}}(\rho))\le \Tr(\rho)$), so the chain is increasing but bounded.

\paragraph{Defining the loop as a limit.}
Because $\{\den{W^{(n)}}(\rho)\}_{n\ge 0}$ is a non-decreasing sequence of positive operators
bounded in trace, it has a least upper bound in the Löwner order. CoqQ uses monotone convergence
principles on ordered Hilbert spaces to justify that the pointwise limit exists and is again a
well-defined super-operator. We therefore \emph{define}:
\[
  \den{W}(\rho)
  \;:=\;
  \lim_{n\to\infty}\ \den{W^{(n)}}(\rho).
\]
Operationally, $\den{W}(\rho)$ is the total (subnormalized) output state contributed by all
terminating executions of the loop, with the traces accounting for the total termination probability.

Since each approximant captures termination within $n$ iterations, the scalar
$\Tr(\den{W^{(n)}}(\rho))$ is the probability mass of terminating within $n$ iterations
from initial state $\rho$. Consequently, the limit
\[
  \Tr(\den{W}(\rho))
  \;=\;
  \lim_{n\to\infty}\Tr(\den{W^{(n)}}(\rho))
\]
is the total probability of termination of the loop on input $\rho$.

\paragraph{Relation to domain-theoretic least fixed points.}
A classical domain-theoretic semantics interprets a while-loop as a least fixed point of a functional
on a CPO of state transformers. In the present setting, define the measurement super-operators
\[
  \mathcal{M}_b(\rho) := M_b^{(s)}\,\rho\,\bigl(M_b^{(s)}\bigr)^\dagger
  \qquad (b\in\{0,1\}).
\]
Then the loop equation informally reads:
\[
  \den{W}(\rho)
  \;=\;
  \mathcal{M}_0(\rho)\;+\;\den{W}\!\bigl(\den{C}(\mathcal{M}_1(\rho))\bigr),
\]
i.e.\ ``stop on outcome $0$, otherwise run $C$ and loop again''.
We will show that the limit-of-approximants definition above coincides with the standard
least-fixed-point semantics, using a lemma that connects suprema of increasing chains of quantum
operations with pointwise limits of super-operators.

\paragraph{Why super-operators?}
There are two complementary reasons to take \emph{super-operators} as the
semantic objects of qwhile commands.

First, super-operators are the mathematically natural level at which quantum programs compose.
Every primitive command we allow---local unitary evolution, initialization (reset), and measurement
followed by classical control---acts on density operators by an operation of the form
\[
\rho \ \longmapsto\ \sum_i E_i\,\rho\,E_i^\dagger,
\]
possibly with a single Kraus operator (unitaries) or with a sum over branches (measurements and
conditionals). This transformation is \emph{linear} in $\rho$ and is completely determined by the
operators describing the physical action on the subsystem. Linearity encodes the
principle that a mixed input state (a probabilistic mixture of preparations) is transformed by the
program into the corresponding mixture of outputs. As a result, sequential composition and program
equivalence reduce to ordinary algebra of linear maps:
\[
\den{C_1;C_2} \;=\; \den{C_2}\circ \den{C_1}.
\]

Second, super-operators provide the right domain for giving a robust semantics of \textsf{while}.
To interpret loops we need an order structure and a notion of limits for increasing chains of
approximants. The Löwner order on positive operators lifts pointwise to super-operators, turning
program denotations into an ordered space where one can define
\[
\den{\mathsf{while}\ M'_s=1\ \mathsf{do}\ C\ \mathsf{od}}
\;=\;
\lim_{n\to\infty}\ \den{W^{(n)}},
\]
and justify that this limit exists and yields a well-defined linear transformer.
This ``limit-of-approximants'' viewpoint also aligns with the standard least-fixed-point semantics of
while-programs, but avoids additional domain-theoretic machinery in the first presentation.

Finally, although the intended semantic objects are \emph{quantum operations} (completely positive,
trace-nonincreasing maps on density operators), it is technically convenient to define denotations
at the broader super-operator level: it allows uniform linear-algebraic reasoning and smooth
compositional definitions. One then proves, as a separate theorem, that every $\den{C}$ generated by
the qwhile syntax is indeed a quantum operation.



\begin{theorem}[]
    For every command $C$, the denotation $\den{C}$ is a quantum operation 
    (i.e., completely positive and trace-nonincreasing), and the loop 
    semantics defined by the limit of syntactic approximants coincides with
    the standard domain-theoretic semantics (least fixed point in the CPO of quantum
    operations).
\end{theorem}



\begin{lemma}[]\label{lem:comp}
If $E,F\in \QO(H)$, then $E\circ F\in \QO(H)$.
\end{lemma}
\begin{proof}
\emph{CP:} $(E\circ F)\otimes I=(E\otimes I)\circ (F\otimes I)$, and composition of positive maps is positive.\\
\emph{TNI:} for $A\sqsupseteq 0$,
\[
\Tr\bigl((E\circ F)(A)\bigr)=\Tr\bigl(E(F(A))\bigr)\le \Tr(F(A))\le \Tr(A).
\]
\end{proof}


\begin{lemma}[]\label{lem:kraus}
Kraus form implies quantum operation. That is, if
\[
E(A)=\sum_i K_i A K_i^\dagger
\qquad\text{and}\qquad
\sum_i K_i^\dagger K_i \sq I,
\]
then $E$ is CP and TNI (hence $E\in \QO(H)$).
\end{lemma}

\begin{proof}
\emph{CP:} immediate, since $E\otimes I$ has Kraus operators $\{K_i\otimes I\}$.\\
\emph{TNI:} for $A\sqsupseteq 0$,
\[
\Tr(E(A))
=\sum_i \Tr(K_i A K_i^\dagger)
=\sum_i \Tr(K_i^\dagger K_i A)
=\Tr\!\Bigl(\Bigl(\sum_i K_i^\dagger K_i\Bigr)A\Bigr)
\le \Tr(A).
\]
\end{proof}


\begin{theorem}\label{thm:A}
For every qwhile command $C$, $\den{C}\in \QO(H)$.
\end{theorem}

\begin{proof}
Recall that $\QO(H)$ denotes the set of \emph{quantum operations} on $H$, i.e.\ super-operators
$\mathcal{E}:L(H)\to L(H)$ that are completely positive (CP) and trace-nonincreasing (TNI).
We prove the claim by structural induction on the syntax of $C$.

\medskip\noindent
\textit{Case 1: $C=\abort$.}
By definition, $\den{\abort}(\rho)=0$ for all $\rho$. The zero map is CP (it preserves positivity
even after tensoring with identities) and TNI since $\Tr(0)=0\le \Tr(\rho)$. Hence
$\den{\abort}\in\QO(H)$.

\medskip\noindent
\textit{Case 2: $C=\Skip$.}
$\den{\Skip}(\rho)=\rho$. The identity super-operator is CP and trace-preserving (hence TNI).
So $\den{\Skip}\in\QO(H)$.

\medskip\noindent
\textit{Case 3: $C=C_1;C_2$.}
By the denotational clause for sequencing,
$ \den{C_1;C_2} \;=\; \den{C_2}\circ \den{C_1}. $, and by the lemma \ref{lem:comp}, we have $\den{C_1;C_2}\in\QO(H)$.

\medskip\noindent
\textit{Case 4: $C=\mathsf{init}\ \rho_s$.}
Let $s\subseteq\Reg$ and write $R:=\comp{s}$, so $H\cong H_s\otimes H_R$.
By definition,
\[
\den{\mathsf{init}\ \rho_s}(\rho) \;=\; \rho_s \otimes \Tr_s(\rho).
\]
We show this map is a quantum operation by exhibiting Kraus operators.

Fix an orthonormal basis $\{\ket{i}\}_i$ of $H_s$ and a spectral decomposition
\[
\rho_s=\sum_j p_j \ket{\psi_j}\!\bra{\psi_j},
\qquad
p_j\ge 0,\ \ \sum_j p_j=1.
\]
Define operators $K_{ij}\in L(H_s\otimes H_R)$ by
\[
K_{ij}:=\sqrt{p_j}\,\bigl(\ket{\psi_j}\!\bra{i}\otimes \Id_{H_R}\bigr).
\]
Then
\[
K_{ij}^\dagger K_{ij}
=
p_j\bigl(\ket{i}\!\bra{i}\otimes \Id_{H_R}\bigr),
\]
so
\[
\sum_{i,j} K_{ij}^\dagger K_{ij}
=
\Bigl(\sum_j p_j\Bigr)\Bigl(\sum_i \ket{i}\!\bra{i}\Bigr)\otimes \Id_{H_R}
=
\Id_{H_s}\otimes \Id_{H_R}
=
\Id.
\]
Hence the induced map is CP and trace-preserving (thus TNI). Moreover, for any $\rho$,
\begin{align*}
\sum_{i,j} K_{ij}\rho K_{ij}^\dagger
&=\sum_{i,j} p_j(\ket{\psi_j}\!\bra{i}\otimes \Id)\,\rho\,(\ket{i}\!\bra{\psi_j}\otimes \Id)\\
&=\sum_j p_j\Bigl(\ket{\psi_j}\!\bra{\psi_j}\otimes \sum_i (\bra{i}\otimes \Id)\rho(\ket{i}\otimes \Id)\Bigr)\\
&=\Bigl(\sum_j p_j \ket{\psi_j}\!\bra{\psi_j}\Bigr)\otimes \Tr_s(\rho)\\
&=\rho_s\otimes \Tr_s(\rho).
\end{align*}
Thus $\den{\mathsf{init}\ \rho_s}\in\QO(H)$.

\medskip\noindent
\textit{Case 5: $C=\mathsf{apply}\ U_s$.}
Let $U_s$ be unitary on $H_s$. By definition,
\[
\den{\mathsf{apply}\ U_s}(\rho)
\;=\;
U_s^{(s)}\,\rho\,\bigl(U_s^{(s)}\bigr)^\dagger,
\]
where $U_s^{(s)}:=U_s\otimes \Id_{H_{\comp{s}}}$ is the cylindrical extension.
This is a Kraus map with a single Kraus operator $U_s^{(s)}$, and
$\bigl(U_s^{(s)}\bigr)^\dagger U_s^{(s)}=\Id$, so it is CP and trace-preserving (hence TNI).
Therefore $\den{\mathsf{apply}\ U_s}\in\QO(H)$.

\medskip\noindent
\textit{Case 6: $C=\mathsf{if}\ (\square m.\ M_s=m\rightarrow C_m)\ \mathsf{fi}$.}
Let $M_s=\{(m,M_m)\}_{m\in\mathsf{Out}(M_s)}$ be a measurement on $H_s$
with $\sum_m M_m^\dagger M_m=\Id_{H_s}$.
By definition,
\[
\den{\mathsf{if}\ (\square m.\ M_s=m\rightarrow C_m)\ \mathsf{fi}}(\rho)
\;=\;
\sum_{m\in\mathsf{Out}(M_s)}
\den{C_m}\!\Bigl(M_m^{(s)}\,\rho\,\bigl(M_m^{(s)}\bigr)^\dagger\Bigr).
\]

\smallskip\noindent
\emph{CP.} For each $m$, the map $\rho\mapsto M_m^{(s)}\rho(M_m^{(s)})^\dagger$ is CP (single Kraus
operator). By IH, $\den{C_m}$ is CP, and composition preserves CP. Finally, a finite sum of CP maps
is CP. Hence the whole denotation is CP.

\smallskip\noindent
\emph{TNI.} Let $\rho\sqsupseteq 0$. Using IH (each $\den{C_m}$ is TNI),
\[
\Tr\!\Bigl(\den{C_m}\bigl(M_m^{(s)}\rho(M_m^{(s)})^\dagger\bigr)\Bigr)
\;\le\;
\Tr\!\Bigl(M_m^{(s)}\rho(M_m^{(s)})^\dagger\Bigr).
\]
Summing over $m$ gives
\begin{align*}
\Tr\!\Bigl(\den{\mathsf{if}\cdots\mathsf{fi}}(\rho)\Bigr)
&\le \sum_m \Tr\!\Bigl(M_m^{(s)}\rho(M_m^{(s)})^\dagger\Bigr)\\
&= \Tr\!\Bigl(\Bigl(\sum_m (M_m^{(s)})^\dagger M_m^{(s)}\Bigr)\rho\Bigr)\\
&= \Tr(\rho),
\end{align*}
since $\sum_m (M_m^{(s)})^\dagger M_m^{(s)}=\Id$ (cylindrical extension preserves completeness).
Therefore the denotation is TNI, hence in $\QO(H)$.

\medskip\noindent
\textit{Case 7: $C=\mathsf{while}\ M'_s=1\ \mathsf{do}\ C_{\mathrm{body}}\ \mathsf{od}$.}
Let $M'_s=\{M_0,M_1\}$ be a two-outcome measurement on $H_s$.
As in Section~\ref{subsec:while-approximants}, define syntactic approximants $W^{(n)}$ by
\[
W^{(0)} := \abort,
\qquad
W^{(n+1)} :=
\mathsf{if}\ (\square b.\ M'_s=b \rightarrow D_b)\ \mathsf{fi},
\quad
D_0:=\Skip,\ \ D_1:=C_{\mathrm{body}};W^{(n)}.
\]
The loop denotation is defined by the increasing limit
\[
\den{\mathsf{while}\ M'_s=1\ \mathsf{do}\ C_{\mathrm{body}}\ \mathsf{od}}
\;:=\;
\bigsqcup_{n\ge 0}\ \den{W^{(n)}}
\;=\;
\lim_{n\to\infty}\den{W^{(n)}}.
\]

\smallskip\noindent
\emph{-each approximant denotes a quantum operation.}
We show $\den{W^{(n)}}\in\QO(H)$ by induction on $n$.
For $n=0$, $W^{(0)}=\abort$ and Case~1 applies.
For the step $n\mapsto n+1$, the command $W^{(n+1)}$ is an \textsf{if} whose branches use only
$\Skip$ and $C_{\mathrm{body}};W^{(n)}$. By IH and Cases~2--6, the denotations of these constructs are
in $\QO(H)$, hence $\den{W^{(n+1)}}\in\QO(H)$.

\smallskip\noindent
\emph{-the approximant chain is nondecreasing.}
Write $E_n:=\den{W^{(n)}}$.
Intuitively, $W^{(n)}$ accounts for executions that terminate within at most $n$ loop-iterations,
so allowing $n{+}1$ iterations can only add additional terminating probability mass. Formally, one
shows (by induction on $n$) that $E_n\sq E_{n+1}$ in the pointwise Löwner order on super-operators:
\[
E_n\sq E_{n+1}\quad:\!\!\iff\quad \forall \rho\sqsupseteq 0,\ E_n(\rho)\sq E_{n+1}(\rho).
\]


\begin{lemma}[]\label{lem:approx-nondecreasing}
Let $W \equiv \mathsf{while}\ M'_s=1\ \mathsf{do}\ C_{\mathrm{body}}\ \mathsf{od}$ with
$M'_s=\{M_0,M_1\}$ a two-outcome measurement on $H_s$, and let $W^{(n)}$ be the syntactic
approximants as above. Write $E_n:=\den{W^{(n)}}$. Then $E_n\sq E_{n+1}$ for all $n\ge 0$, i.e.
\[
\forall n\ge 0,\qquad
\forall \rho\sqsupseteq 0,\qquad
E_n(\rho)\ \sq\ E_{n+1}(\rho).
\]
\end{lemma}

\begin{proof}
Fix $n\ge 0$. We first rewrite $E_{n+1}$ in a convenient unfolded form.

\smallskip\noindent
\emph{Unfolding.}
By the denotational clause for \textsf{if} (specialized to outcomes $b\in\{0,1\}$) and using
$\den{\Skip}(\rho)=\rho$ and $\den{C_1;C_2}=\den{C_2}\circ \den{C_1}$, we have for all $\rho$:
\begin{align}
E_{n+1}(\rho)
&=\den{D_0}\!\Bigl(M_0^{(s)}\rho\bigl(M_0^{(s)}\bigr)^\dagger\Bigr)
 \;+\;
 \den{D_1}\!\Bigl(M_1^{(s)}\rho\bigl(M_1^{(s)}\bigr)^\dagger\Bigr)\nonumber\\
&= M_0^{(s)}\rho\bigl(M_0^{(s)}\bigr)^\dagger
 \;+\;
 \bigl(E_n\circ \den{C_{\mathrm{body}}}\bigr)\!\Bigl(M_1^{(s)}\rho\bigl(M_1^{(s)}\bigr)^\dagger\Bigr).\label{eq:Enplus1-unfold}
\end{align}
Define two auxiliary (positive) maps on inputs $\rho\sqsupseteq 0$:
\[
T(\rho) := M_0^{(s)}\rho\bigl(M_0^{(s)}\bigr)^\dagger,
\qquad
B(\rho) := \den{C_{\mathrm{body}}}\!\Bigl(M_1^{(s)}\rho\bigl(M_1^{(s)}\bigr)^\dagger\Bigr).
\]
Then \eqref{eq:Enplus1-unfold} becomes simply
\begin{equation}\label{eq:Enplus1-TB}
E_{n+1}(\rho)=T(\rho)+E_n\bigl(B(\rho)\bigr).
\end{equation}
Similarly,
\begin{equation}\label{eq:Enplus2-TB}
E_{n+2}(\rho)=T(\rho)+E_{n+1}\bigl(B(\rho)\bigr).
\end{equation}

\smallskip\noindent
\emph{Positivity facts.}
If $\rho\sqsupseteq 0$, then $T(\rho)\sqsupseteq 0$ because $\rho\mapsto M_0^{(s)}\rho(M_0^{(s)})^\dagger$
is a (single-Kraus) CP map. Also $B(\rho)\sqsupseteq 0$ because $\rho\mapsto M_1^{(s)}\rho(M_1^{(s)})^\dagger$
is CP and $\den{C_{\mathrm{body}}}$ is CP by the outer induction hypothesis on program structure
(the body is a strict subcommand of the loop).

\smallskip\noindent
\emph{Induction on $n$.}
We prove $E_n\sq E_{n+1}$ for all $n$ by induction on $n$.

\smallskip\noindent
\underline{Base $n=0$.}
$E_0=\den{W^{(0)}}=\den{\abort}$, hence $E_0(\rho)=0$ for all $\rho$.
Since $E_1(\rho)\sqsupseteq 0$ for all $\rho\sqsupseteq 0$ (by \eqref{eq:Enplus1-TB} with $n=0$),
we have $E_0(\rho)=0\sq E_1(\rho)$.

\smallskip\noindent
\underline{Step.}
Assume $E_n\sq E_{n+1}$, i.e.\ for all $X\sqsupseteq 0$, $E_n(X)\sq E_{n+1}(X)$.
We must show $E_{n+1}\sq E_{n+2}$.

Fix any $\rho\sqsupseteq 0$. Then $B(\rho)\sqsupseteq 0$, so applying the induction hypothesis to
$X=B(\rho)$ yields
\[
E_n\bigl(B(\rho)\bigr)\ \sq\ E_{n+1}\bigl(B(\rho)\bigr).
\]
Adding the same positive operator $T(\rho)$ to both sides preserves the Löwner order, hence
\[
T(\rho)+E_n\bigl(B(\rho)\bigr)\ \sq\ T(\rho)+E_{n+1}\bigl(B(\rho)\bigr).
\]
Using \eqref{eq:Enplus1-TB} and \eqref{eq:Enplus2-TB}, this is exactly
$E_{n+1}(\rho)\sq E_{n+2}(\rho)$.

\smallskip\noindent
Thus $E_n\sq E_{n+1}$ for all $n\ge 0$.
\end{proof}

\smallskip\noindent
\emph{-increasing limits preserve being a quantum operation.}
Since each $E_n$ is CP and TNI and $(E_n)$ is nondecreasing, the pointwise supremum
$E:=\bigsqcup_{n\ge 0} E_n$ exists (in finite dimension) and remains CP and TNI.
Equivalently, $E$ is a quantum operation and is the least upper bound of the chain.
\begin{lemma}[Limit of a CP-increasing chain is a quantum operation]\label{lem:limit-QO}
Let $(E_n)_{n\ge 0}$ be a sequence in $\QO(H)$ such that for every $n$,
\[
E_{n+1}-E_n \text{ is completely positive.}
\]
(Equivalently, $E_n \sq_{\mathrm{cp}} E_{n+1}$ in the CP-order.)
Then:
\begin{enumerate}[label=(\roman*),leftmargin=*]
\item For every $\rho\sqsupseteq 0$, the increasing sequence $(E_n(\rho))_{n\ge 0}$ has a supremum (equivalently a limit)
in the Löwner order; define
\[
E(\rho)\;:=\;\bigsqcup_{n\ge 0} E_n(\rho)\;=\;\lim_{n\to\infty}E_n(\rho)\qquad(\rho\sqsupseteq 0).
\]
\item The resulting linear map $E:L(H)\to L(H)$ belongs to $\QO(H)$ (i.e.\ $E$ is CP and TNI).
\item Moreover, $E$ is the least upper bound of $(E_n)$ with respect to the pointwise order $\sq$:
\[
E=\bigsqcup_{n\ge 0} E_n\quad\text{in the sense that}\quad
(\forall n,\ E_n\sq E)\ \text{and}\ (\forall F,\ (\forall n,\ E_n\sq F)\Rightarrow E\sq F).
\]
\end{enumerate}
\end{lemma}

\begin{proof}
Since $H$ is finite-dimensional, we use the Choi--Jamio\l kowski representation.

\smallskip\noindent
\emph{(1) Choi matrices and monotonicity.}
Fix an orthonormal basis $\{\ket{i}\}_{i=1}^d$ of $H$ and let
\[
\ket{\Omega}:=\sum_{i=1}^d \ket{i}\otimes \ket{i}\in H\otimes H.
\]
For a super-operator $\mathcal{E}:L(H)\to L(H)$ define its Choi matrix
\[
J(\mathcal{E}) \;:=\; (\mathcal{E}\otimes \Id)\bigl(\ket{\Omega}\!\bra{\Omega}\bigr)\ \in L(H\otimes H).
\]
We use the standard facts:
\begin{itemize}[leftmargin=*]
\item[] $\mathcal{E}$ is CP iff $J(\mathcal{E})\sqsupseteq 0$.
\item[] $\mathcal{E}$ is TNI iff $\Tr_1(J(\mathcal{E}))\sq \Id$, where $\Tr_1$ traces out the first factor.
\item[] The map $\mathcal{E}\mapsto J(\mathcal{E})$ is linear and injective.
\end{itemize}
Write $J_n:=J(E_n)$. Since each $E_n$ is CP, $J_n\sqsupseteq 0$.

By assumption, $E_{n+1}-E_n$ is CP. Hence $(E_{n+1}-E_n)\otimes \Id$ is CP (in particular positive),
and therefore
\[
J_{n+1}-J_n
=
\bigl((E_{n+1}-E_n)\otimes \Id\bigr)\bigl(\ket{\Omega}\!\bra{\Omega}\bigr)\sqsupseteq 0.
\]
Thus $(J_n)$ is increasing in the Löwner order: $J_n\sq J_{n+1}$.

\smallskip\noindent
\emph{(2) Boundedness and existence of the limit Choi matrix.}
Each $E_n$ is TNI, so $\Tr_1(J_n)\sq \Id$.
Taking trace and using $\Tr(J_n)=\Tr(\Tr_1(J_n))$, we obtain the uniform bound
\[
\Tr(J_n)=\Tr(\Tr_1(J_n))\le \Tr(\Id)=d\qquad\text{for all }n.
\]
Hence $(J_n)$ is an increasing sequence of PSD operators with uniformly bounded trace.
In finite dimension, such a sequence converges: there exists $J\sqsupseteq 0$ such that
\[
J=\lim_{n\to\infty}J_n
\qquad\text{and}\qquad
J_n\sq J\ \text{ for all }n.
\]

\smallskip\noindent
\emph{(3) Define $E$ by the limit Choi matrix; $E$ is CP and TNI.}
By injectivity of the Choi representation, there is a unique linear map $E:L(H)\to L(H)$ with
\[
J(E)=J.
\]
Since $J\sqsupseteq 0$, Choi's theorem implies $E$ is CP.
Also partial trace is continuous, so
\[
\Tr_1(J)=\Tr_1\!\Bigl(\lim_{n\to\infty}J_n\Bigr)
=\lim_{n\to\infty}\Tr_1(J_n)\sq \Id,
\]
because each $\Tr_1(J_n)\sq \Id$. Thus $E$ is TNI, hence $E\in\QO(H)$.

\smallskip\noindent
\emph{(4) Pointwise supremum and least-upper-bound property in the order $\sq$.}
First, $E$ is an upper bound for $(E_n)$ in the pointwise order $\sq$.
Indeed, $J-J_n\sqsupseteq 0$ implies $E-E_n$ is CP (hence positive), so for all $\rho\sqsupseteq 0$,
\[
E_n(\rho)\sq E(\rho),
\]
i.e.\ $E_n\sq E$.

Next, we identify $E(\rho)$ as the pointwise supremum. Since $J_n\to J$ and the inverse Choi map is a linear
isomorphism between finite-dimensional vector spaces, it is continuous; therefore for every $\rho\sqsupseteq 0$,
\[
E_n(\rho)\longrightarrow E(\rho).
\]
Because $(E_n(\rho))_{n\ge 0}$ is increasing in the Löwner order (as $E_{n+1}-E_n$ is positive),
its limit equals its supremum:
\[
E(\rho)=\bigsqcup_{n\ge 0}E_n(\rho)\qquad(\rho\sqsupseteq 0).
\]

Finally, $E$ is the least upper bound in the pointwise order.
Let $F$ be any upper bound, i.e.\ $E_n\sq F$ for all $n$. Then for each $\rho\sqsupseteq 0$,
\[
E_n(\rho)\sq F(\rho)\quad\forall n \ \Longrightarrow\ 
\bigsqcup_{n\ge 0}E_n(\rho)\sq F(\rho).
\]
Using $E(\rho)=\bigsqcup_{n\ge 0}E_n(\rho)$, we obtain $E(\rho)\sq F(\rho)$ for all $\rho\sqsupseteq 0$, i.e.\ $E\sq F$.
Hence $E=\bigsqcup_{n\ge 0}E_n$ in the sense of the pointwise order $\sq$.
\end{proof}
\smallskip\noindent
Therefore the loop denotation (defined as this supremum) lies in $\QO(H)$.

\end{proof}


\subsection{The domain-theoretic setup}

The paper contrasts its loop semantics (a limit of syntactic approximants) with the classic
domain-theoretic semantics (a least fixed point in a CPO of quantum operations), and then claims
that the two coincide. We now prove this coincidence in our setting.

The ``classic'' approach assumes $(\QO(H),\sq)$ is a CPO and that the loop functional is
Scott-continuous, so that least fixed points exist and can be obtained via the Kleene construction.

In our concrete finite-dimensional setting, we do not need to assume these abstract facts.
We already proved the key concrete statement we need: for any nondecreasing $\omega$-chain of
quantum operations, the pointwise supremum exists and yields a quantum operation
(Lemma~\ref{lem:limit-QO}), and this supremum agrees with the corresponding operator limit.

\medskip\noindent
\textbf{Bottom element.}
Let $\bot$ be the zero super-operator:
\[
  \bot(\rho):=0.
\]
Then $\bot\sq E$ for every positive map $E$, hence $\bot$ is the least element of $(\QO(H),\sq)$.


Fix a loop command in qwhile notation:
\[
  W\ \equiv\ \mathsf{while}\ M'_s=1\ \mathsf{do}\ C_{\mathrm{body}}\ \mathsf{od},
\]
where $M'_s=\{M_0,M_1\}$ is a two-outcome measurement on subsystem $s$.
(Outcome $0$ terminates; outcome $1$ continues.)

Define a functional $\Phi:\QO(H)\to \QO(H)$ by
\[
  \Phi(X)(\rho)
  \;:=\;
  M_0^{(s)}\,\rho\,\bigl(M_0^{(s)}\bigr)^\dagger
  \;+\;
  X\!\Bigl(\den{C_{\mathrm{body}}}\bigl(M_1^{(s)}\,\rho\,\bigl(M_1^{(s)}\bigr)^\dagger\bigr)\Bigr).
\]
This is the denotation of the one-step unfolding of the loop: measure the guard; if outcome is $0$,
stop; if outcome is $1$, execute the body and then ``recur'', with the recursive call represented by
the parameter $X$.

\begin{lemma}\label{lem:phiQO}
$\Phi$ maps $\QO(H)$ to $\QO(H)$.
\end{lemma}

\begin{proof}
Fix $X\in\QO(H)$. We show $\Phi(X)\in\QO(H)$.

\smallskip\noindent
\emph{CP.}
The map $\rho\mapsto M_0^{(s)}\rho(M_0^{(s)})^\dagger$ is CP (single Kraus operator).
The map $\rho\mapsto \den{C_{\mathrm{body}}}(M_1^{(s)}\rho(M_1^{(s)})^\dagger)$ is a composition of
quantum operations (apply a Kraus map, then $\den{C_{\mathrm{body}}}$), hence is CP.
Composing with $X$ preserves CP, and a finite sum of CP maps is CP.
Thus $\Phi(X)$ is CP.

\smallskip\noindent
\emph{TNI.}
Let $\rho\sqsupseteq 0$. Using that $X$ is TNI and $\den{C_{\mathrm{body}}}$ is TNI
(Theorem~\ref{thm:A}), we obtain
\begin{align*}
\Tr(\Phi(X)(\rho))
&=
\Tr\!\Bigl(M_0^{(s)}\rho(M_0^{(s)})^\dagger\Bigr)
+
\Tr\!\Bigl(X\bigl(\den{C_{\mathrm{body}}}(M_1^{(s)}\rho(M_1^{(s)})^\dagger)\bigr)\Bigr)\\
&\le
\Tr\!\Bigl(M_0^{(s)}\rho(M_0^{(s)})^\dagger\Bigr)
+
\Tr\!\Bigl(\den{C_{\mathrm{body}}}(M_1^{(s)}\rho(M_1^{(s)})^\dagger)\Bigr)\\
&\le
\Tr\!\Bigl(M_0^{(s)}\rho(M_0^{(s)})^\dagger\Bigr)
+
\Tr\!\Bigl(M_1^{(s)}\rho(M_1^{(s)})^\dagger\Bigr)\\
&=
\Tr\!\Bigl(\bigl((M_0^{(s)})^\dagger M_0^{(s)}+(M_1^{(s)})^\dagger M_1^{(s)}\bigr)\rho\Bigr)\\
&=
\Tr(\rho),
\end{align*}
since $(M_0^{(s)})^\dagger M_0^{(s)}+(M_1^{(s)})^\dagger M_1^{(s)}=\Id$ by measurement completeness
(cylindrical extension preserves the identity).
So $\Phi(X)$ is TNI. Hence $\Phi(X)\in\QO(H)$.
\end{proof}

\paragraph{Approximants are iterates of $\Phi$ from $\bot$.}

Let $(W^{(n)})_{n\ge 0}$ be the syntactic approximants for the loop $W$ defined earlier:
\[
W^{(0)} := \abort,
\qquad
W^{(n+1)} :=
\mathsf{if}\ (\square b.\ M'_s=b \rightarrow D_b)\ \mathsf{fi},
\quad
D_0:=\Skip,\ \ D_1:=C_{\mathrm{body}};W^{(n)}.
\]
Write $E_n:=\den{W^{(n)}}\in\QO(H)$.

\begin{lemma}\label{lem:iterates}
$E_0=\bot$ and $E_{n+1}=\Phi(E_n)$ for all $n\ge 0$.
\end{lemma}

\begin{proof}
By denotation of $\abort$, $E_0=\den{\abort}=\bot$.

For the step, expand $E_{n+1}$ using the denotational clauses for \textsf{if}, \textsf{skip}, and
sequencing:
\begin{align*}
E_{n+1}(\rho)
&=
\den{\Skip}\!\Bigl(M_0^{(s)}\rho(M_0^{(s)})^\dagger\Bigr)
+
\den{C_{\mathrm{body}};W^{(n)}}\!\Bigl(M_1^{(s)}\rho(M_1^{(s)})^\dagger\Bigr)\\
&=
M_0^{(s)}\rho(M_0^{(s)})^\dagger
+
E_n\!\Bigl(\den{C_{\mathrm{body}}}(M_1^{(s)}\rho(M_1^{(s)})^\dagger)\Bigr)\\
&=
\Phi(E_n)(\rho).
\end{align*}
Thus $E_{n+1}=\Phi(E_n)$.
In particular, for all $n$,
$ E_n=\Phi^n(\bot). $
\end{proof}


The classic domain-theoretic semantics of the loop is the least fixed point
\[
  \lfp(\Phi),
\]
taken in $(\QO(H),\sq)$. The Kleene construction says that, for a Scott-continuous (or at least
$\omega$-continuous) $\Phi$,
\[
  \lfp(\Phi)=\bigsqcup_{n\ge 0}\Phi^n(\bot).
\]
We now show directly that the supremum of the
approximant denotations is a least fixed point of $\Phi$.

\begin{lemma}\label{lem:lfp}
Let $E_n:=\den{W^{(n)}}$ and define
\[
  E \;:=\; \bigsqcup_{n\ge 0} E_n.
\]
Then $E$ is a fixed point of $\Phi$ and is the least fixed point of $\Phi$ in $(\QO(H),\sq)$.
\end{lemma}

\begin{proof}

By Lemma~\ref{lem:approx-nondecreasing}, $E_n\sq E_{n+1}$ for all $n$.
By Lemma~\ref{lem:limit-QO}, the supremum $E=\bigsqcup_{n\ge 0}E_n$ exists and belongs to $\QO(H)$.

\smallskip\noindent
\emph{-$E$ is a fixed point of $\Phi$.}
Using Lemma~\ref{lem:iterates}, $E_{n+1}=\Phi(E_n)$ for all $n$.
Taking suprema of both sides yields
\[
  \bigsqcup_{n\ge 0} E_{n+1}
  \;=\;
  \bigsqcup_{n\ge 0} \Phi(E_n).
\]
Since shifting an $\omega$-chain does not change its supremum,
$\bigsqcup_{n\ge 0} E_{n+1}=\bigsqcup_{n\ge 0} E_n=E$.
So it remains to show
\[
  \Phi\!\Bigl(\bigsqcup_{n\ge 0}E_n\Bigr)
  \;=\;
  \bigsqcup_{n\ge 0}\Phi(E_n).
\]
Fix $\rho\sqsupseteq 0$. Write
\[
  T(\rho):=M_0^{(s)}\rho(M_0^{(s)})^\dagger,
  \qquad
  B(\rho):=\den{C_{\mathrm{body}}}(M_1^{(s)}\rho(M_1^{(s)})^\dagger).
\]
Then
\[
  \Phi(X)(\rho)=T(\rho)+X\bigl(B(\rho)\bigr).
\]
Using the pointwise definition of supremum in the Löwner order,
\[
  E\bigl(B(\rho)\bigr)
  \;=\;
  \bigsqcup_{n\ge 0} E_n\bigl(B(\rho)\bigr).
\]
Adding the fixed positive operator $T(\rho)$ preserves suprema:
\[
  T(\rho)+E\bigl(B(\rho)\bigr)
  \;=\;
  \bigsqcup_{n\ge 0}\bigl(T(\rho)+E_n\bigl(B(\rho)\bigr)\bigr).
\]
But $T(\rho)+E_n(B(\rho))=\Phi(E_n)(\rho)$, hence
\[
  \Phi(E)(\rho)=\bigsqcup_{n\ge 0}\Phi(E_n)(\rho).
\]
Since this holds for all $\rho\sqsupseteq 0$, we conclude $\Phi(E)=\bigsqcup_{n\ge 0}\Phi(E_n)$.
Combining with $\bigsqcup_{n\ge 0}\Phi(E_n)=\bigsqcup_{n\ge 0}E_{n+1}=E$, we obtain $\Phi(E)=E$.

\smallskip\noindent
\emph{Step 3: $E$ is the least fixed point.}
Let $F\in\QO(H)$ be any fixed point, $\Phi(F)=F$.
Since $\bot\sq F$ and $\Phi$ is monotone, we have by induction
\[
  E_n=\Phi^n(\bot)\ \sq\ \Phi^n(F)=F
  \qquad\text{for all }n.
\]
Taking suprema over $n$ yields $E=\bigsqcup_{n\ge 0}E_n\sq F$.
Therefore $E$ is the least fixed point of $\Phi$.
\end{proof}

\end{document}