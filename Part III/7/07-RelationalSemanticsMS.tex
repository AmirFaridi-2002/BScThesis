\documentclass[11pt]{article}

\usepackage[a4paper,margin=1in]{geometry}
\usepackage{amsmath,amssymb,amsthm,mathtools,mathrsfs}
\usepackage{braket,stmaryrd}
\usepackage{enumitem}
\usepackage{xcolor}
\usepackage{tikz-cd}

\newcommand{\LL}{\mathcal{L}}
\newcommand{\Tr}{\mathrm{tr}}
\newcommand{\Id}{\mathbf{1}}
\newcommand{\QO}{\mathrm{QO}}
\newcommand{\sq}{\sqsubseteq}
\newcommand{\den}[1]{\llbracket #1 \rrbracket}
\newcommand{\Var}{\mathsf{Var}}
\newcommand{\Val}{\mathsf{Val}}
\newcommand{\AExp}{\mathsf{AExp}}
\newcommand{\BExp}{\mathsf{BExp}}
\newcommand{\Cmd}{\mathsf{Cmd}}
\newcommand{\post}{\operatorname{post}}
\newcommand{\wlp}{\operatorname{wlp}}
\newcommand{\Guard}{\operatorname{Guard}}
\newcommand{\lfp}{\operatorname{lfp}}
\newcommand{\supp}{\operatorname{supp}}
\newcommand{\ok}{\mathsf{ok}}
\newcommand{\er}{\mathsf{er}}
\newcommand{\true}{\mathsf{true}}
\newcommand{\false}{\mathsf{false}}
\newcommand{\error}{\mathsf{error}}
\newcommand{\abort}{\mathsf{abort}}
\newcommand{\assume}{\mathsf{assume}}
\newcommand{\assert}{\mathsf{assert}}
\newcommand{\ifthenelse}[3]{\mathsf{if}\ #1\ \mathsf{then}\ #2\ \mathsf{else}\ #3}
\newcommand{\whiledo}[2]{\mathsf{while}\ #1\ \mathsf{do}\ #2}
\newcommand{\nat}{\mathsf{nat}}
\newcommand{\infer}[3][]{\frac{#2}{#3}\,#1}
\newcommand{\Skip}{\mathsf{skip}}
\newcommand{\States}{\Sigma}
\newcommand{\Pow}{\mathscr{P}}

\renewcommand{\braket}[2]{\langle #1 | #2 \rangle}

\newcommand{\Reg}{\mathsf{Reg}}
\newcommand{\Done}{\downarrow}
\newcommand{\Conf}[2]{\langle #1,\, #2\rangle}
\newcommand{\step}[1]{\xrightarrow{#1}}
\newcommand{\ptr}{\operatorname{tr}}
\newcommand{\cyl}{\operatorname{cyl}}
\newcommand{\Dens}{\mathcal{D}}
\newcommand{\Dsub}[1]{\mathcal{D}_{\le 1}(#1)}
\newcommand{\comp}[1]{\overline{#1}}
\newcommand{\Mset}{\mathsf{Mset}}
\newcommand{\mset}[1]{\{\!\{#1\}\!\}}  
\newcommand{\msum}{\operatorname{Sum}} 
\newcommand{\Reach}{\operatorname{Reach}}
\newcommand{\Out}{\operatorname{Out}}
\newcommand{\Mix}{\operatorname{Mix}}




\theoremstyle{plain}
\newtheorem{theorem}{Theorem}[section]
\newtheorem{lemma}[theorem]{Lemma}
\newtheorem{claim}[theorem]{Claim}

\theoremstyle{definition}
\newtheorem{definition}[theorem]{Definition}

\theoremstyle{remark}
\newtheorem{remark}[theorem]{Remark}

\begin{document}

\begin{center}
{\Large Quantum Programming Languages and Semantics}\\[0.5em]
\end{center}
\setcounter{section}{6}



\section{Relational Semantics}\label{sec:relational}

This section introduces a second semantic view of qwhile programs that is particularly suited
for under-approximate reasoning.
In the previous section, we interpreted each command $C$ as a single state transformer
$\den{C} : \LL(H)\to \LL(H)$ (a super-operator), so probabilistic branching is already
``collapsed'' by summation into one output partial state (a mixture). This is exactly what makes
denotational semantics compositional and algebraic, but it also hides the structure of
\emph{which} final states are reachable along \emph{which} (classically-controlled) execution paths.

The QIL paper therefore introduces a relational/collecting semantics that maps an input state
to a \emph{collection} of reachable terminal states (separately for normal vs.\ abnormal exits),
and only afterwards forms the probabilistic mixture of that collection.


For qwhile we will keep in mind the following three, mutually consistent, semantic perspectives:

\begin{enumerate}[leftmargin=*, label=$\roman*$]
\item \textit{Operational semantics}:
configurations $\Conf{C}{\rho}$ step to $\Conf{C'}{\rho'}$, and measurement-induced branching
is represented by multiple possible successor configurations with subnormalized branch states.

\item \textit{Super-operator denotation}:
$\den{C}$ maps an input partial state $\rho$ to a single output partial state $\den{C}(\rho)$,
obtained by summing (mixing) the contributions of all branches. 

\item \textit{Relational/collecting denotation}:
for each exit condition $\epsilon\in\{\ok,\er\}$ we assign a multiset of reachable terminal states
$\den{C}_\epsilon(\rho)$; the mixture of reachable states is then obtained by summing that multiset.
\end{enumerate}

A set of reachable outcomes forgets multiplicities: two different execution paths might end in the
same terminal state, and for some compositional constructions (notably sequential composition) it
is technically convenient to remember this information.

Let $A$ be any universe (here, $A=\Dsub{H}$).
A \emph{multiset} over $A$ is a function $\nu : A \to \nat$; $\nu(a)$ is the multiplicity of $a$.
We write $\Mset(A)$ for the set of multisets over $A$.
The empty multiset is the constant-zero function, also written $0$ when no confusion arises.
We use the multiset braces $\mset{a}$ for the singleton multiset: $\mset{a}(a)=1$ and
$\mset{a}(b)=0$ for $b\neq a$.

Multiset union is pointwise addition:
\[
(\nu_1 \uplus \nu_2)(a) \;:=\; \nu_1(a)+\nu_2(a).
\]
If $R_1,R_2 : \Dsub{H}\to \Mset(\Dsub{H})$ are multiset-valued semantic functions, we lift $\uplus$
pointwise:
\[
(R_1 \uplus R_2)(\rho) \;:=\; R_1(\rho)\uplus R_2(\rho).
\]
Any multiset transformer $R:\Dsub{H}\to \Mset(\Dsub{H})$ can be seen as a \emph{weighted relation}:
\[
R(\rho)(\rho') = n
\quad\text{means}\quad
\text{``$\rho'$ is reachable from $\rho$ with multiplicity $n$.''}
\]
When we only care about reachability (existence), we can drop multiplicities and define
\[
\Reach_R(\rho) \;:=\; \{\rho' \in \Dsub{H} : R(\rho)(\rho')>0\}.
\]
This is the ``relational'' reading: a program induces (for each exit condition) a relation between
input states and terminal states.

Sequential composition of programs corresponds to composing multiset transformers in the obvious
``run the first, then the second'' way.
Given $R_1,R_2 : \Dsub{H}\to \Mset(\Dsub{H})$, define $(R_1\circ R_2):\Dsub{H}\to\Mset(\Dsub{H})$ by
\[
(R_1\circ R_2)(\rho)(\rho'')
\;:=\;
\sum_{\rho'\in\Dsub{H}} R_1(\rho)(\rho')\cdot R_2(\rho')(\rho'').
\]
Intuitively: each intermediate state $\rho'$ contributes all of $R_2(\rho')$,
and multiplicities multiply along a path and add across alternative paths.

We distinguish normal termination ($\ok$) from abnormal termination
($\er$). In the collecting semantics, this distinction is made at the semantic type level:
\[
\den{C}_\epsilon \;:\; \Dsub{H} \longrightarrow \Mset(\Dsub{H})
\qquad (\epsilon\in\{\ok,\er\}),
\]
where $\den{C}_\ok(\rho)$ is the multiset of terminal states reachable by executions that exit normally,
and $\den{C}_\er(\rho)$ is the multiset of terminal states reachable by executions that exit due to $\error$.

\emph{impossible executions} are represented by the zero operator $0\in \Dsub{H}$.
Thus, when a particular exit condition is impossible, the denotation returns the singleton multiset
$\mset{0}$ (which contributes no probability mass to mixtures).
In this section, unlike the previous one, we use $\error$ instead of $\abort$ to distinguish between normal and erroneous termination.


\subsubsection*{Inductive definition of $\den{C}_\epsilon$}

Throughout, $\rho\in \Dsub{H}$, and all subsystem-local operators are interpreted via cylindrical
extension, e.g.\ $M_m^{(s)} := M_m\otimes \Id_{H_{\comp{s}}}$ under the fixed identification
$H\cong H_s\otimes H_{\comp{s}}$.

\paragraph{Skip and error.}
\[
\den{\Skip}_\ok(\rho) := \mset{\rho},
\qquad
\den{\Skip}_\er(\rho) := \mset{0},
\]
\[
\den{\error}_\ok(\rho) := \mset{0},
\qquad
\den{\error}_\er(\rho) := \mset{\rho}.
\]
Thus, $\Skip$ produces the current state as a normal outcome, while $\error$ produces the current
state as an abnormal outcome; the other exit condition is impossible and is therefore mapped to
the $0$-state multiset.

\paragraph{Initialization and unitary application.}
Initialization always terminates normally and resets subsystem $s$ to $\rho_s$:
\[
\den{\mathsf{init}\ \rho_s}_\ok(\rho) := \mset{\rho_s \otimes \Tr_s(\rho)},
\qquad
\den{\mathsf{init}\ \rho_s}_\er(\rho) := \mset{0}.
\]
Unitary application also always terminates normally:
\[
\den{\mathsf{apply}\ U_s}_\ok(\rho) :=
\mset{U_s^{(s)}\,\rho\,\bigl(U_s^{(s)}\bigr)^\dagger},
\qquad
\den{\mathsf{apply}\ U_s}_\er(\rho) := \mset{0}.
\]

\paragraph{Conditionals.}
Let $M_s=\{(m,M_m)\}_{m\in\Out(M_s)}$ be a measurement on $H_s$.
For $\epsilon\in\{\ok,\er\}$,
\[
\den{\mathsf{if}\ (\square m.\ M_s=m\rightarrow C_m)\ \mathsf{fi}}_\epsilon(\rho)
\;:=\;
\biguplus_{m\in\Out(M_s)}
\den{C_m}_\epsilon\!\Bigl(M_m^{(s)}\,\rho\,\bigl(M_m^{(s)}\bigr)^\dagger\Bigr).
\]
Each outcome $m$ produces a (subnormalized) post-measurement state; the conditional then
collects the terminal states produced by the selected branch program $C_m$.
This collecting clause is the multiset analogue of the super-operator clause in
the previous section, where one sums branch \emph{states} rather than collecting them. 

\paragraph{Sequencing.}
Sequential composition splits into two cases, reflecting the short-circuiting behavior of $\error$ in
the operational semantics:
\[
\den{C_1;C_2}_\ok \;:=\; \den{C_2}_\ok \circ \den{C_1}_\ok,
\]
\[
\den{C_1;C_2}_\er
\;:=\;
\den{C_1}_\er \uplus \bigl(\den{C_2}_\ok \circ \den{C_1}_\er\bigr).
\]
The first line says: to exit normally, $C_1$ must exit normally and then $C_2$ must exit normally.
The second line says: an abnormal exit happens either immediately in $C_1$ (in which case $C_2$
is not executed), or after $C_1$ exits normally and $C_2$ exits abnormally.
\paragraph{While loops.}
Let
\[
W \;\equiv\; \mathsf{while}\ M'_s=1\ \mathsf{do}\ C\ \mathsf{od},
\qquad
M'_s=\{M_0,M_1\},
\]
where outcome $0$ terminates the loop and outcome $1$ executes the body and repeats.
We define the collecting semantics of $W$ \emph{directly} as the multiset
union of all \emph{finite} numbers of loop iterations, rather than introducing a separate syntactic
approximant program.

\smallskip\noindent
Treat each guard outcome as a (singleton) multiset transformer on partial states:
\[
\mathcal{M}_b(\rho)\;:=\;\mset{\,M_b^{(s)}\,\rho\,\bigl(M_b^{(s)}\bigr)^\dagger\,}
\qquad (b\in\{0,1\}),
\]
where $M_b^{(s)}:=M_b\otimes \Id_{H_{\comp{s}}}$ is the cylindrical extension.

\smallskip\noindent
For any multiset transformer $R:\Dsub{H}\to\Mset(\Dsub{H})$, define its iterates by
\[
R^{0}(\rho)\;:=\;\mset{\rho},
\qquad
R^{n+1}\;:=\;R^{n}\circ R,
\]
where $\circ$ is the composition on multiset transformers defined earlier.

\smallskip\noindent
\textbf{Normal termination ($\ok$).}
A normal terminating execution of the loop consists of some number $n$ of \emph{continue} rounds
(outcome $1$, then executing $C$ normally), followed by one \emph{stop} round (outcome $0$).
Therefore we define
\[
\den{W}_{\ok}
\;:=\;
\biguplus_{n\in\nat}
\Bigl(\,(\mathcal{M}_{1}\circ \den{C}_{\ok})^{n}\circ \mathcal{M}_{0}\Bigr),
\]
i.e.\ for each $\rho\in\Dsub{H}$,
\[
\den{W}_{\ok}(\rho)
\;=\;
\biguplus_{n\in\nat}
\Bigl((\mathcal{M}_{1}\circ \den{C}_{\ok})^{n}\circ \mathcal{M}_{0}\Bigr)(\rho).
\]
The summand $n=0$ corresponds to immediate termination (observe $0$ right away); the summand
$n=1$ corresponds to one continue-iteration (observe $1$, run $C$, then observe $0$); and so on.

\smallskip\noindent
\textbf{Abnormal termination ($\er$).}
An abnormal termination of the loop can only come from an abnormal termination of the body $C$.
Such an execution consists of $n$ successful continue-iterations (outcome $1$ and $C$ exits normally),
followed by one continue outcome $1$ whose subsequent body execution exits abnormally.
Thus we define
\[
\den{W}_{\er}
\;:=\;
\biguplus_{n\in\nat}
\Bigl(\,(\mathcal{M}_{1}\circ \den{C}_{\ok})^{n}\circ (\mathcal{M}_{1}\circ \den{C}_{\er})\Bigr),
\]
i.e.\ for each $\rho$,
\[
\den{W}_{\er}(\rho)
\;=\;
\biguplus_{n\in\nat}
\Bigl((\mathcal{M}_{1}\circ \den{C}_{\ok})^{n}\circ (\mathcal{M}_{1}\circ \den{C}_{\er})\Bigr)(\rho).
\]

\smallskip\noindent
These definitions match the ``collect all reachable terminal states'' idea: each $n$ picks out
the terminal states reachable by executions with exactly $n$ completed iterations, and the outer
$\biguplus_{n\in\nat}$ collects (with multiplicity) the outcomes from \emph{all} finite iteration counts.
The mixture of reachable states used later is then obtained by summing the resulting multiset.


\paragraph{Mixture of reachable states.}
For correctness-style reasoning, the super-operator denotation $\den{C}(\rho)$ is already a single
partial density operator representing the post-execution ensemble.
For incorrectness-style reasoning, however, we want first to \emph{collect} reachable outcomes and
only then to form the ensemble mixture, because the under-approximate triple is phrased in terms
of ``the mixture of reachable states''.

Let $\nu\in \Mset(\Dsub{H})$ be a multiset of partial density operators.
Define its \emph{mixture} (sum) by
\[
\msum(\nu) \;:=\; \sum_{\sigma\in\Dsub{H}} \nu(\sigma)\cdot \sigma.
\]
Because every $\sigma\succeq 0$, the partial sums form an increasing sequence in the Löwner order.
In finite dimension, and for the multisets produced by qwhile programs, this sum converges and is
well-defined.

The denotational clauses for qwhile only generate (at most) countably many distinct terminal
states from any fixed input, since each step introduces only finitely many measurement outcomes
and looping generates at most countably many finite unrollings.
Moreover, each terminal state is subnormalized, and the total trace mass of terminating outcomes
is bounded by $\Tr(\rho)$ (intuitively: one cannot create probability mass).
Therefore the trace of $\msum(\nu)$ is finite and bounded, which rules out divergence of the sum
in finite dimension.

\paragraph{Mixture of reachable states for a program.}
For a command $C$ and exit condition $\epsilon\in\{\ok,\er\}$, define the \emph{exit-conditioned mixture}
\[
\Mix_\epsilon(C,\rho) \;:=\; \msum\!\bigl(\den{C}_\epsilon(\rho)\bigr)\ \in\ \Dsub{H}.
\]
Thus $\Mix_\ok(C,\rho)$ is the mixture of all normally terminating reachable states,
and $\Mix_\er(C,\rho)$ is the mixture of all abnormally terminating reachable states.

Sometimes we also write the total terminating mixture (forgetting the exit label) as
\[
\Mix(C,\rho) \;:=\; \Mix_\ok(C,\rho)\;+\;\Mix_\er(C,\rho).
\]
This is the most direct quantum analogue of ``collect all terminating outcomes and mix them.''

The super-operator semantics collapses branching by
summing the branch contributions immediately (e.g.\ in conditionals, it sums the post-measurement
branch states after applying branch denotations). 
By contrast, the collecting semantics first returns the multiset of branch results and only then
sums them via $\msum$.

These two perspectives coincide once we ``forget'' the exit condition and interpret $\error$ as
non-returning (abort) at the super-operator level:
\[
\den{C}(\rho)\;=\;\Mix_\ok(C,\rho).
\]
Intuitively, $\den{C}(\rho)$ is exactly the mixture of all \emph{normally} reachable outcomes,
because the super-operator semantics accounts only for returned post-states (and treats
abnormal termination as contributing $0$).

Later, the incorrectness triple will not talk about individual terminal states directly;
instead, it talks about predicates that are under-approximated by the mixture of reachable states.
The support-based under-approximation relation is what
lets one extract ``there exists a bad outcome'' information from a single mixed state:
even if bad states occur with small probability mass, they can still contribute support and thus be
detectable by an under-approximate specification.

\end{document}